\documentclass[12pt]{article}


\usepackage[utf8]{inputenc}
\usepackage{sbc-template}
\usepackage{graphicx}
\usepackage{tabularx}
\newcolumntype{C}{>{\centering\arraybackslash}X}    % centered flexible tabularX column
\usepackage{url}
\renewcommand{\UrlFont}{\ttfamily\small}
\urldef{\TheRepoUrl}\url{https://github.com/nico-ralf-ii-fpuna/paper}


\sloppy


\title{
    Anomaly-based Web Application Firewall using
    HTTP-specific features and One-Class SVM
}
\author{
    Nico Epp\inst{1},
    Ralf Funk\inst{1},
    Cristian Cappo\inst{1}
}
\address{
    Facultad Politécnica -- Universidad Nacional de Asunción\\
    San Lorenzo -- Paraguay
    \email{
        \{nicoeppfriesen,ralffunk0\}@gmail.com, ccappo@pol.una.py
    }
}


\begin{document} 
    \maketitle

    \begin{abstract}
        Vulnerabilities in web applications pose great risks because they
        can be exploited by malicious attackers through the Internet.
        Web Application Firewalls placed in front of these applications
        can help to minimize these risks.
        In this paper, we present such a firewall based on anomaly detection
        that aims to detect anomalous HTTP requests using One Class SVM
        classifier.
        Our work uses expert knowledge about the HTTP request structure to
        build feature extraction methods that improve the detection rates.
        We include a link to the online repository that contains the code
        of our implementation for the purpose of reproducibility and
        extensibility.
    \end{abstract}

    % the command '\include' adds pagebreaks, so we use '\input' instead
    \section{Introduction}

Web applications have become mainstream in the last decade.
Their vulnerabilities pose a great risk because since the applications
are accessible through the Internet, these vulnerabilities can easily be
exploited by malicious attackers.
Web Application Firewalls (WAF) can be placed in front of these
applications to minimize the risk of attacks \cite{torranoGimenez2015study}.

According to the detection method, a WAF can be either \textit{misuse-based},
when in looks for known attack signatures,
or \textit{anomaly-based}, when it aims to differentiate between normal and
anomalous HTTP requests, where the anomalies can include malicious attacks
\cite{torranoGimenez2015study}.
In order to accomplish this differentiation task, the WAF needs to recognize
characteristics about the individual HTTP requests.
This kind of WAF has two phases, namely training and detection.
During the training phase, it extracts models from observed normal requests.
During the detection phase, it compares the incoming traffic to these models
and flags requests as anomalous if they deviate significantly from normal
requests previously seen.
A great strength of this approach lies with the fact that the WAF only needs
to be retrained if there are changes in the applications it protects
and not because of the appearance of new attacks \cite{kruegel2003anomaly}.

For the anomaly detection process of the WAF, one possible strategy is to
face it as a classification problem with a single positive class,
denominated One Class Classification (OCC) problem \cite{khan2009survey}.
This way the normal HTTP requests belong to this single positive class and
all the anomalies will be categorized as not belonging to this class.
This OCC approach has the advantage of not needing labeled anomalous data
for training, only normal requests are required.
One methodology to solve OCC problems is to employ tools from the area of
Machine Learning, one of those being the \textit{One-Class} SVM, which has
been used with great success for this kind of problems \cite{khan2009survey}.

For the WAF to be able to recognize the characteristics of HTTP requests,
these features need to be expressed in a format that the employed tools
can process.
For instance, the \textit{One-Class} SVM cannot work directly with raw data
and it requires a data preprocessing step.
This step transforms the features of the input requests as numeric feature
vectors, which are then passed to the \textit{One-Class} SVM.
Expert knowledge about HTTP requests and current attacks can help in selecting
the most useful features for the classification process between normal and
anomalous requests \cite{kruegel2003anomaly}.

In this article we present a WAF based on anomaly detection that uses features
selected with expert knowledge about HTTP requests and also uses
\textit{One-Class} SVM classifier to detect anomalous requests.
In Section \ref{chap:related_works} we present some related works, in Section
\ref{chap:proposal} we describe our proposed WAF, Section \ref{chap:results}
contains our test results and we finish this paper with some conclusions and
ideas for future work in Section \ref{chap:conclusions}.


\section{Related works}
\label{chap:related_works}

In \cite{kruegel2003anomaly} and \cite{kruegel2005multi} the authors use
expert knowledge about the structure of HTTP requests in their anomaly
detection system.
They describe how the query string of an HTTP request consists of an ordered
list of $n$ pairs of parameters and their corresponding values and how
they use stochastic methods to build anomaly models of the values for
each of the parameters. Some of the models they use are the length of the
values, their character distribution, structural inference, parameter
presence, among others.
For example, one of their models describes the string length of the different
values that appear for the parameter \textit{username} in the URL
\textit{/login}.

The \textit{One-Class} SVM has been used successfully in many areas,
including text classification, face recognition, spam detection, anomaly
detection, among others \cite{khan2009survey}.
The authors in \cite{parhizkar2015oc} use this classifier to successfully
detect anomalies in web traffic, but their approach differs from ours in
that they extract a fixed number of features of the requests.


\section{Proposal}
\label{chap:proposal}

In this section, we describe the four main parts that make up our WAF,
namely a routing step, a data preprocessing step, a classification step
and lastly a response step.
Our implementation was done in the programming language \textit{Python}
using the Machine Learning library \textit{scikit-learn} and the code
is accessible in our online repository under \TheRepoUrl.


\subsection{Routing step}

Since requests to the same URL show a greater similarity to each other
than to requests to other URLs, our routing step groups the incoming
requests by URL and HTTP method. This way during the training phase the
anomaly models can be built independently for each group, resulting in a
more precise description of the normal requests within each group.
Consequently, during the detection phase there are more accurate models of
normal requests available, which help to identify requests that are
anomalous within their corresponding groups, even though they would be
considered normal in other groups.


\subsection{Data preprocessing step}

Our data preprocessing step extracts a vector of numeric features from
each HTTP request, which is then passed to the classification step.
We use feature extraction methods that yield a total of 10 numeric features.
Our first feature represents the length or character count, the second
counts the number of digits, the third the number of alphabetical
characters and the fourth counts the number of characters that are not
alphanumerical. We based these features on \cite{kruegel2003anomaly} and
added some extensions.
The next five features represent five bins of a method called character
distribution, also based on \cite{kruegel2003anomaly}.
Our tenth and last feature represents the entropy according to Shannon
\cite{encyMathEntropy}, an idea taken from \cite{nguyen2011application}.

Our feature extraction methods are applied on the whole request,
including HTTP method, URL, query string, headers and body.
Additionally, we employ the previously mentioned approach used in
\cite{kruegel2003anomaly} and apply our extraction methods on each of the
parameter values in the query string. We extend this approach to also
analyze the parameter values in the body of the requests, if there are any.

The obtained feature vector that represents a request will have
$m \times (1 + n)$ features.
Here $m$ is the number of features returned by our feature extraction
methods for each value.
The $1$ represents the whole request and $n$ is the number of parameters
that are present in the query string and body.
Its important to note that $n$ might be different for each group of URL
and HTTP method.
During the training phase, all the different parameters from a set of
requests are listed and given a fixed order for building the feature vector.
This way, during the detection phase our WAF extracts the features of the
values whose parameters have been seen in the training phase, assuring that
their position within the vector is preserved.
For example, if during training our WAF gets POST request which all have
only the parameters \textit{username} and \textit{password} in its body,
then $m = 10$ and $n = 2$, resulting in each request being represented by
a vector of 30 features.

We are aware that our analysis of parameter values during the detection
phase is only applied to parameters observed in training.
This gives way to the risk of an intruder including an attack inside the
value of a previously unseen parameter.
But since our feature extraction methods are also applied on the whole
request, these attacks can still be detected.


\subsection{Classification step}

During the training phase, our WAF trains one \textit{One-Class} SVM
classifier for each group of URL and HTTP method, using the numeric
feature vectors produced by the previous step.
In this high dimensional vector space, the classifier tries to find boundaries
to the regions in which the feature vectors of normal request reside,
drawing these borders as tight as possible to exclude future vectors of
anomalous requests, but loose enough to still include future normal requests
not seen in the training phase \cite{perdisci2006using}.
This way the WAF obtains a model, a trained classifier, that generalizes
the characteristics of normal HTTP requests within each group.

During the detection phase, the incoming requests are routed to their
corresponding group, the feature extraction methods are applied and the
trained classifier for that group checks if this new feature vector
falls inside the established boundaries, marking it as normal if it does.
All other requests will be denoted as anomalous.

The \textit{One-Class} SVM takes a few parameters that influence its
behavior.
One of these parameters, denominated $nu$ or $\nu$, is used to set an upper
bound to the fraction of training requests that may fall outside of the
established boundaries. This gives the classifier some flexibility when
drawing the borders in order to achieve higher generalization capabilities
and also achieves robustness in case of some anomalies in the training
data \cite{scholkopf2001estimating}.
Sometimes it can be difficult to find boundaries in the given vector space,
so the classifier can use a so called kernel function to simulate a
higher dimensionality in which it is easier to draw the separating borders.
We tested three kernels, namely the linear, the polynomial and the
\textit{Radial Basis Function} (RBF) kernel. We obtained the best
classification results with the last one.
The RBF kernel also has a parameter called $gamma$ or $\gamma$ that
influences the shape of the boundaries \cite{tran2004one}.


\subsection{Response step}

Our WAF can be configured to respond in different ways to the classification
results. If the HTTP requests are considered normal, no further action
needs to be taken apart from forwarding them to their intended destination.
However, if our classification step considered a request as anomalous,
the WAF logs the result and optionally can also block that request,
preventing possible attacks from even reaching the protected web applications.
Our implementation could be extended to take different actions, for
example sending alarms about anomalies to the people responsible
for the system.


\section{Experiments and results}
\label{chap:results}

For the quantitative evaluation of the performance of our WAF we used the
public data sets CSIC 2010 \cite{csic2010dataset} and CSIC TORPEDA 2012
\cite{torpeda2012dataset}, which contain labeled samples of simulated
HTTP traffic to an e-commerce web application.
Given that our WAF groups the requests by URL and HTTP method, we grouped
the available data from both data sets by these criteria and selected
those groups that had more than 100 samples of each of the two categories,
normal and anomalous requests.
This left us with 18 groups, totaling 40,130 normal and 42,444 anomalous
HTTP requests.


\subsection{Detection effectiveness test}

In a first test, we used the aforementioned data to measure the detection
effectiveness of our WAF.
To demonstrate the added value of analyzing the parameter values,
we tested two different scenarios.
In the first scenario, we applied our feature extraction methods only to
the whole request, obtaining a fixed number of 10 features.
The second scenario included the analysis of the parameter values, yielding
a different number of features for each group, as described in the previous
section.
Additionally, for the selection of the values for the parameters $\nu$
and $\gamma$ of the classifier within each group, we tested a few values
in the range $[0.0001: 0.1]$ and chose those values that gave the best
result in each individual group.

To compare the results we employ the $F_{1}$ score, which uses the number
of true positives ($TP$), false positives ($FP$) and false negatives ($FN$)
obtained in the classification process.
This measure can be expressed as $ F_{1} = \frac{2TP}{2TP + FP + FN} $
and values closer to 1 indicate better results.

Table \ref{tbl:test_1_results} shows that the second scenario improves
the average $F_{1}$ from $0.83$ to $0.90$ compared to the first,
demonstrating the value of using the analysis of parameter values.
We can also observe that the second scenario achieves a perfect score
of $1.00$ for one of the 18 groups, while the first scenario does not
get that high for any of the groups.

\begin{table}[t]
    \centering
    \caption{Results of detection effectiveness test for all 18 groups}
    \label{tbl:test_1_results}

    \begin{tabularx}{\linewidth}{|c|C|C|}
        \hline
        Scenario                               & average $F_{1}$ & best $F_{1}$ \\ \hline
        Using only whole request               & $0.83 \pm 0.09$ & $0.95$       \\ \hline
        Including analysis of parameter values & $0.90 \pm 0.08$ & $1.00$       \\ \hline
    \end{tabularx}
\end{table}


\subsection{Response time analysis}

Our second test aimed to quantify the effects our WAF has on the response
time of the web applications it protects. To that end, we set up a simple
application and timed the intervals between sending a request and receiving
the response. This test was run in three different scenarios; in one the
requests went directly to the application, in the second the traffic was
routed through our WAF but with detection disabled, and in the third
scenario with enabled detection.
The second and third scenarios show an increase of 2 and 3 ms respectively
when compared to the scenario without the WAF.


\subsection{Training time analysis}

In our third test, we analyzed how the training time of our WAF relates to
the amount of training samples used. Since this measurement is influenced
by the number of extracted features, each group may have different durations
and so we took the highest number to get an upper bound.
We observed that the training time per request stays almost constant at
3 ms per request in our test setup for up to 10,000 requests.
This shows that the training time has an almost linear relationship to the
amount of training data.


\section{Conclusions}
\label{chap:conclusions}

The WAF we presented in this article uses the \textit{One-Class} SVM
classifier to tackle the task of detecting anomalous HTTP requests in
order to protect web applications from possible attacks.

Following and extending the ideas obtained from the works of other authors,
we built feature extraction methods that take advantage of the structure of
HTTP requests to represent these requests with vectors which contain more
useful information.
Our test results show a significant improvement in the detection results
when using this knowledge about the request structure, specifically the
analysis of parameter values.

One contribution of this paper is the adaptation and extension of the ideas
in \cite{kruegel2003anomaly} to make feature extraction methods useful for
Machine Learning tools, in our case specifically the \textit{One-Class} SVM.
As another contribution, we leave a link in Section \ref{chap:proposal}
to the online repository that contains the code of our implementation,
which includes the WAF and the tests mentioned in this paper, so that
other authors can reproduce our results and have a starting point for
further research.
Our implementation shows that it is fast enough to be used as a WAF, even
though the minimization of the processing time was not our primary goal
and further optimizations could be made.

Future works could look into finding additional features that help to
detect anomalous requests.
Another research area is other classification tools to be used instead
of the \textit{One-Class} SVM to solve the OCC task.
Additionally, our work lacks a method for automatic selection of the
parameters of the classifier and further inquiries can be made in that
direction.


    \bibliography{p_references}
    \bibliographystyle{sbc}
\end{document}
